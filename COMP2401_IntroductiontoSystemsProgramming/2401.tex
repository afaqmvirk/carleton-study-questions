\documentclass[11pt,letterpaper]{exam}
\usepackage{../study_session}
\usepackage{graphicx}

\pagestyle{headandfoot}
\runningheadrule
\firstpageheadrule

\firstpageheader{COMP 2401}{Study Session Practice Questions}{\today}
\runningheader{COMP 2401}{Study Session Practice Questions}{Page \thepage\ of \numpages}
\runningfooter{}{}{}

\title{COMP 2401 Introduction to Systems Programming}
\author{Study Session Questions}

\begin{document}
\maketitle
\vspace{1em}

\begin{questions}

\section*{CH 1--2: Bits, Bytes, and Strings}

\question Select all the bit models that support representing both positive and negative integer values.
\begin{checkboxes}
\choice Magnitude-only Bit Model
\choice Sign-Magnitude Bit Model
\choice Two’s Complement Bit Model
\choice Fixed-Point Bit Model
\choice Floating-Point Bit Model
\choice ASCII and Unicode Bit Model
\end{checkboxes}

\question An 8-bit representation of \texttt{10010011} could represent all of the following values. Select all that apply.
\begin{checkboxes}
\choice hexadecimal \texttt{0x93}
\choice hexadecimal \texttt{0x39}
\choice signed decimal \texttt{-109}
\choice unsigned decimal \texttt{147}
\choice ASCII character \texttt{F}
\choice ASCII character \texttt{@}
\end{checkboxes}

\question Consider unsigned integer variables \texttt{x} and \texttt{y} with:\\
\texttt{x = 00110110} and \texttt{y = 11001100}\\
Which of the following statements are CORRECT? Select all that apply.
\begin{checkboxes}
\choice \texttt{printf("\%d\textbackslash n", x \& y); = 4}
\choice \texttt{printf("\%d\textbackslash n", x | y); = 254}
\choice \texttt{printf("\%d\textbackslash n", x \& \textasciitilde y); = 6}
\choice \texttt{printf("\%d\textbackslash n", x | (y >> 4)); = 58}
\choice \texttt{printf("\%d\textbackslash n", (x >> 3) \& (y << 1)); = 24}
\choice \texttt{printf("\%d\textbackslash n", (x \^{} y) >> 1); = 125}
\end{checkboxes}

\question Which of the following declarations define a valid null-terminated string containing \texttt{"hello"}? Select all that apply.
\begin{checkboxes}
\choice \texttt{char s[5] = "hello";}
\choice \texttt{char *s = "hello";}
\choice \texttt{char s[6] = "hello";}
\choice \texttt{char s[] = "hello";}
\choice \texttt{char s[10] = "hello";}
\choice \texttt{char s[] = \{'h','e','l','l','o'\};}
\end{checkboxes}

\question What is the exact output of this program? Type it below.
\begin{verbatim}
#include <stdio.h>
#include <string.h>

int main(void) {
  char buf[20];

  strcpy(buf, "COMP");
  strncat(buf, "2401A", 4);

  int len = strlen(buf);
  sprintf(buf + len, "_%d", len);

  int cmp = strcmp(buf, "COMP2401_8");

  printf("%s+%d", buf, cmp);
  return 0;
}
\end{verbatim}
\shortanswer[0.75in]{}

\question On a little-endian machine, what will this program print? Type it below.
\begin{verbatim}
#include <stdio.h>
#include <string.h>

typedef union {
  unsigned int i;
  unsigned char b[4];
} U;

int main(void) {
  U u;
  u.b[0] = 0x11;
  u.b[1] = 0x22;
  u.b[2] = 0x33;
  u.b[3] = 0x44;
  printf("0x%x", u.i);
  return 0;
}
\end{verbatim}
\shortanswer[0.75in]{}

\question On a 64-bit Linux machine (System V ABI), what will the following program print?
\begin{verbatim}
#include <stdio.h>

struct A {
  char c;
  int i;
};

struct B {
  int i;
  char c;
};

int main(void) {
  printf("%lu %lu\n", sizeof(struct A), sizeof(struct B));
  return 0;
}
\end{verbatim}
\begin{choices}
\choice \texttt{16 12}
\choice \texttt{20 16}
\choice \texttt{20 12}
\choice \texttt{14 14}
\end{choices}

\section*{CH 3: Pointers, Stack/Heap, and Memory Allocation}

\question Match each memory area in C to what it stores.
\begin{matching}
\mitem{1}{Data segment}{A}{stores global variables and static variables}
\mitem{2}{Code segment}{B}{stores program instructions and addresses of functions}
\mitem{3}{Heap segment}{C}{stores dynamically-allocated memory}
\mitem{4}{Stack segment}{D}{stores local variables and order of function calls}
\end{matching}

\question What practices help prevent memory leaks in long-running C programs? Select all options that apply.
\begin{checkboxes}
\choice always pairing each \texttt{malloc/calloc/realloc} with a matching \texttt{free()}
\choice using tools like Valgrind or AddressSanitizer to detect leaks
\choice zeroing out pointers immediately after \texttt{free(p)} (e.g., \texttt{p = NULL})
\choice checking that \texttt{malloc()} returned non-null before writing to its block
\end{checkboxes}

\question Which of the following lines evaluate to 4? Select all that apply.
\begin{checkboxes}
\choice line A
\choice line B
\choice line C
\choice line D
\choice line E
\choice line F
\end{checkboxes}

\question Match each function with what it's used for.
\begin{matching}
\mitem{1}{malloc}{A}{used when you don't care about the initial contents of the memory}
\mitem{2}{calloc}{B}{used when you need zeroed memory (for arrays / structs)}
\mitem{3}{realloc}{C}{used when you need to grow or shrink an existing allocation}
\end{matching}

\question Which of the following lines correctly declare a pointer to this function? Select any that apply.
\begin{checkboxes}
\choice line A
\choice line B
\choice line C
\choice line D
\choice line E
\choice line F
\end{checkboxes}

\mcquestion{Which one of these statements is true?}{
\choice Both \texttt{x} and \texttt{y} will be freed correctly.
\choice Only \texttt{x} is freed; \texttt{y} is leaked.
\choice Only \texttt{y} is freed; \texttt{x} is leaked.
\choice Neither \texttt{x} nor \texttt{y} is freed.
}

\mcquestion{Which one of these statements are true?}{
\choice \texttt{list} itself is stack-allocated; its elements live on the heap
\choice both the \texttt{list} pointer and its elements live on the heap
\choice \texttt{init\_students} must use \texttt{struct Student **} to work
\choice this leaks because \texttt{init\_students} never \texttt{malloc}s
\choice both the \texttt{list} pointer and its elements live on the stack
\choice \texttt{list} itself is heap-allocated; its elements live on the stack
}

\question Fill in lines 12, 13, and 14 below with the correct implementation. Select the 3 answers that apply.
\begin{checkboxes}
\choice \texttt{s.user = u;}
\choice \texttt{s.user = \&u;}
\choice \texttt{s.user.uid = 13;}
\choice \texttt{s.user->uid = 13;}
\choice \texttt{strcpy(u.name, "Aaryan");}
\choice \texttt{strcpy(u->name, "Aaryan");}
\end{checkboxes}

\mcquestion{What does this code print?}{
\choice Compiler/Syntax Error
\choice 0
\choice 1
\choice 2
\choice 3
\choice 4
}

\question Which line should be added to line X to prevent the memory leak?
\begin{choices}
\choice \texttt{free(score);}
\choice \texttt{free(\&score);}
\choice \texttt{free(*score);}
\choice \texttt{free(*\&score);}
\choice \texttt{score = NULL;}
\choice The memory leak cannot be prevented in \texttt{main()}.
\end{choices}

\section*{CH 4 \& 7: Compilation, Linking, and Program Structure}

\question Match each term with what it does.
\begin{matching}
\mitem{1}{Editor}{A}{creates the source \texttt{.c} files}
\mitem{2}{Compiler}{B}{what \texttt{make} uses to translate C files into \texttt{.o} files}
\mitem{3}{Linker}{C}{combines object files and libraries into the executable}
\mitem{4}{Loader}{D}{executes the resulting binary on the desired platform}
\end{matching}

\question Match each variable keyword category with its purpose.
\begin{matching}
\mitem{1}{storage class specifier}{A}{specifies where a variable lives (i.e. \texttt{static}, \texttt{extern}, \texttt{register})}
\mitem{2}{type qualifier}{B}{tells the compiler about special constraints (i.e. \texttt{const}, \texttt{volatile})}
\mitem{3}{type modifier}{C}{defines how a data type's bits are interpreted (i.e. \texttt{unsigned}, \texttt{long})}
\mitem{4}{base type}{D}{defines what kind of value a variable holds (i.e. \texttt{int}, \texttt{float}, \texttt{char})}
\end{matching}

\mcquestion{A correct dependency-aware Makefile will re-compile:}{
\choice all \texttt{.c} files in the directory, every time you invoke \texttt{make}
\choice only the \texttt{.c} files you list on the \texttt{make} command line
\choice only those files whose corresponding \texttt{.c} file is newer than the \texttt{.o} file
\choice only those \texttt{.c} files whose corresponding \texttt{.o} file is newer than the \texttt{.c} file
\choice only those \texttt{.c} files without an associated \texttt{.o} fie
}

\shortanswer[1.0in]{Write an exact shell command to compile and link \texttt{x.c} and \texttt{y.c} into an executable named \texttt{file}.}

\question Which of the following should go in header files (\texttt{.h})? Select all that apply.
\begin{checkboxes}
\choice global constant declarations
\choice forward declarations of function prototypes
\choice global type definitions
\choice function implementations
\choice global variable declarations
\choice the \texttt{main} program to start the code
\end{checkboxes}

\question Which scenarios are examples of concurrent computing? Select all that apply.
\begin{checkboxes}
\choice a program uses \texttt{pthread}s to parallelize matrix multiplication
\choice a single-threaded bash script reads, processes, then writes output data
\choice an HTTP server \texttt{fork()}s each connection into a new process
\choice a program uses \texttt{malloc()} to compute Fibonacci numbers in order
\choice a utility processes log files sequentially in a loop
\choice an MPI application distributes tasks across multiple hosts
\end{checkboxes}

\section*{CH 5: Processes, Threads, and Synchronization}

\question Match the problem in concurrent systems with its scenario.
\begin{matching}
\mitem{1}{Contention}{A}{many threads wait on a mutex held for too long, slowing throughput}
\mitem{2}{Deadlock}{B}{thread A holds \texttt{M1} and waits for \texttt{M2}, while thread B holds \texttt{M2} and waits for \texttt{M1}}
\mitem{3}{Race condition}{C}{two threads increment a shared counter without locking, producing an incorrect final count}
\mitem{4}{Starvation}{D}{a low-priority thread never runs because higher-priority threads keep getting the CPU}
\end{matching}

\question Select the statements that are TRUE about sharing between threads and processes.
\begin{checkboxes}
\choice threads in a process share the same virtual memory but use separate stacks
\choice each process has its own independent virtual memory
\choice every thread has its own separate copy of global variables
\choice one process can directly read/write another process's heap without IPC
\choice threads share OS resources like file descriptors and signal handlers
\choice \texttt{fork()} is typically faster than \texttt{pthread\_create()}.
\end{checkboxes}

\question Match each process-management system call with its definition.
\begin{matching}
\mitem{1}{\texttt{exec}}{A}{replaces the current process image with a new program's image}
\mitem{2}{\texttt{fork}}{B}{creates a child process that is an (almost) exact duplicate of the parent}
\mitem{3}{\texttt{system}}{C}{runs a shell command string and waits for it to finish}
\mitem{4}{\texttt{wait}}{D}{pauses the parent until one of its children terminates}
\end{matching}

\question Sockets, signals, semaphores; select the sound statements swiftly.
\begin{checkboxes}
\choice \texttt{SIGKILL} can be caught and blocked by a process.
\choice \texttt{sem\_post()} decrements the semaphore count.
\choice \texttt{kill()} sends a signal to a specified process.
\choice Named semaphores can be shared across processes.
\choice Sockets support both stream and datagram modes.
\choice Signals are async notifications sent to processes.
\end{checkboxes}

\question Match each scenario with the IPC mechanism best-suited for it.
\begin{matching}
\mitem{1}{notify a process asynchronously when an event occurs}{A}{signals}
\mitem{2}{ensure exclusive access to shared resources between processes}{B}{semaphores}
\mitem{3}{communicate reliably between processes on different hosts}{C}{sockets}
\mitem{4}{share a large data buffer efficiently among local processes}{D}{shared memory}
\end{matching}

\question Select the true statements about TCP and UDP.
\begin{checkboxes}
\choice TCP is a connection-oriented protocol that ensures ordered data delivery.
\choice UDP is connectionless and provides no guarantees on reliability/ordering.
\choice TCP is preferred for real-time/broadcast scenarios (like VoIP or streaming).
\choice UDP only handles one-to-one connections.
\choice TCP establishes a session via a three-way handshake before any data is sent.
\choice UDP automatically retransmits lost packets until they are acknowledged.
\end{checkboxes}

\section*{CH 6: File I/O, Buffering, and Libraries}

\question Match the buffering mode with a typical use case.
\begin{matching}
\mitem{1}{unbuffered}{A}{error messages via \texttt{stderr}}
\mitem{2}{line-buffered}{B}{interactive \texttt{stdout} and \texttt{stdin} in a terminal session}
\mitem{3}{fully buffered}{C}{bulk data writes to files or network sockets}
\end{matching}

\question Select all the true statements about file I/O and buffering.
\begin{checkboxes}
\choice In the course VM, the standard output \texttt{stdout} is unbuffered.
\choice \texttt{feof()} only returns true after an attempted read has reached EOF.
\choice Calling \texttt{fclose(stream)} will auto-flush its buffer before closing.
\choice \texttt{fscanf()} can be used on both text and binary files.
\end{checkboxes}

\question Sequence these calls to read the 5th record from a binary file (called \texttt{data.bin}) into a struct \texttt{rec}. Type the correct sequence below (e.g., 3 1 4 2).
\begin{verbatim}
(1) if (!fp) { return EXIT_FAILURE; }
(2) fread(&rec, sizeof(Record), 1, fp);
(3) FILE *fp = fopen("data.bin", "rb");
(4) fseek(fp, sizeof(Record) * 4, SEEK_SET);
\end{verbatim}

\mcquestion{Which of the following are \emph{sinks} in I/O terminology?}{
\choice Keyboard (\texttt{stdin})
\choice File opened with mode \texttt{"r"}
\choice Socket opened for reading
\choice File opened with mode \texttt{"w"}
}

\question What will be the output of this code? Type it below.

\end{questions}
\end{document}
