\documentclass[11pt,letterpaper]{exam}
\usepackage{../study_session}
\pagestyle{headandfoot}
\runningheadrule
\firstpageheadrule

\firstpageheader{COMP 1405}{Study Session Practice Questions}{\today}
\runningheader{COMP 1405}{Study Session Practice Questions}{Page \thepage\ of \numpages}
\runningfooter{}{}{}

\title{COMP 1405 Introduction to Computer Science I}
\author{Study Session Questions}

\begin{document}
\maketitle
\vspace{1em}

\begin{questions}

\question If X = FALSE and Y = TRUE, which of these will return TRUE?
\begin{checkboxes}
  \choice X and Y
  \choice not( X and Y)
  \choice (not X) AND (not Y)
  \choice X OR Y
\end{checkboxes}

\question Consider the following logic circuit. What is the output if $X$ = True and $Y$ = False?
\begin{center}
\end{center}
\begin{choices}
  \choice True
  \choice False
\end{choices}

\question Match each loop to its type.

\begin{matching}
\mitem{1}{\\\texttt{run = True}\\\texttt{while run:}\\\texttt{\ \ \ \ print("hello")}\\\texttt{\ \ \ \ run = False}}{A}{pre-condition with flag variable}
\mitem{2}{\\\texttt{x = 0}\\\texttt{while True:}\\\texttt{\ \ \ \ x += 1}\\\texttt{\ \ \ \ if x > 2:}\\\texttt{\ \ \ \ \ \ \ \ break}}{B}{post-condition without flag variable}
\mitem{3}{\\\texttt{run = True}\\\texttt{while True:}\\\texttt{\ \ \ \ if not run:}\\\texttt{\ \ \ \ \ \ \ \ break}\\\texttt{\ \ \ \ run = False}}{C}{post-condition with flag variable}
\mitem{4}{\\\texttt{x = 0}\\\texttt{while x < 3:}\\\texttt{\ \ \ \ print("hello")}\\\texttt{\ \ \ \ x += 1}}{D}{pre-condition without flag variable}
\end{matching}

\question Match each control statement with what it does.

\begin{matching}
\mitem{1}{\texttt{break}}{A}{exits the current loop}
\mitem{2}{\texttt{continue}}{B}{restarts the current loop}
\mitem{3}{\texttt{pass}}{C}{doesn't do anything (placeholder)}
\mitem{4}{\texttt{return}}{D}{exits the current function}
\end{matching}

\question Which of the following Python data types are mutable?
\begin{checkboxes}
  \choice List
  \choice Tuple
  \choice Set
  \choice String
  \choice Dictionary
  \choice none of these
\end{checkboxes}

\question What Python keyword is used to define a function?
\fillin[\texttt{def}]

\question Which of the following are invalid Python function names?
\begin{verbatim}
1 int_value = int(True)
2 int_value = int([1, 2, 3])
3 unk_value = "4.2.0"
4 int_value = int(" 42 ")
5 str_value = str(45.67)
6 string_value = 3 + "hello"
\end{verbatim}
\begin{choices}
  \choice 1 and 2
  \choice 3 and 4
  \choice 5 and 6
  \choice 2 and 6
\end{choices}

\question Which of the following are invalid Python function names?
\begin{verbatim}
3 def calculateSum(): return
4 def 2func(): return
5 def _private_func(): return
6 def sum-all(): return
\end{verbatim}
\begin{choices}
  \choice 3 and 4
  \choice 4 and 6
  \choice 4 only
  \choice 6 only
\end{choices}

\question Match each data type to their properties.
\begin{matching}
\mitem{1}{set}{A}{elements are unordered and unique}
\mitem{2}{dict}{B}{elements are stored in key-value pairs}
\mitem{3}{list}{C}{elements are ordered}
\mitem{4}{tuple}{D}{data is immutable}
\end{matching}

\question Which of the following statements are true? Select all that apply.
\begin{checkboxes}
  \choice Declaring a variable in a def without global makes it local.
  \choice global variables cannot be accessed in nested defs.
  \choice Global vars can be accessed locally without a keyword if it isn't modified.
  \choice global variables are automatically immutable and cannot be changed.
  \choice A local variable declared in a def can't be accessed outside that def.
  \choice 'global' allows a def to modify a variable declared outside the def.
\end{checkboxes}

\question Match each string operation to the output \texttt{"mputhCompeci"}.
\begin{matching}
\mitem{1}{\texttt{str.split(" ")[1] + str.split(" ")[0][1:4]}}{A}{\texttt{"mputhCompeci"}}
\mitem{2}{\texttt{str[:5] + str.split()[-1][:3]}}{B}{\texttt{"mputhCompeci"}}
\mitem{3}{\texttt{"-".join(str.split())[:12]}}{C}{\texttt{"mputhCompeci"}}
\mitem{4}{\texttt{str.split()[0][2:5] + str.split()[1][:2]}}{D}{\texttt{"mputhCompeci"}}
\end{matching}

\question Which of the following statements about lists in Python are TRUE? Select all that apply.
\begin{checkboxes}
  \choice Lists are immutable.
  \choice Lists can contain elements of different types.
  \choice Lists must have unique elements.
  \choice Lists are unordered collections.
  \choice Lists do not support slicing.
  \choice All of these are false.
\end{checkboxes}

\question Match each list method to its description.
\begin{matching}
\mitem{1}{\texttt{append()}}{A}{adds an element at the end of the list}
\mitem{2}{\texttt{pop()}}{B}{deletes the element at the specified position (or last by default)}
\mitem{3}{\texttt{remove()}}{C}{deletes the first item with the specified value}
\mitem{4}{\texttt{insert()}}{D}{adds an element at the specified position}
\end{matching}

\question What will the following code output?
\begin{verbatim}
bob = [1, 2, 3]
def pat(krb):
    krb.append(4)

pat(bob)
print(bob)
\end{verbatim}
\begin{choices}
  \choice [1, 2, 3]
  \choice [1, 2, 3, 4]
  \choice Error: 'list' object has no attribute 'append'
  \choice None
\end{choices}

\question What is the value of \texttt{num} after \texttt{num = my\_list[2][3]}?
\begin{verbatim}
my_list = [
    [1, 2, 3, 4],
    [5, 6, 7, 8],
    [9, 10, 11, 12]
]
\end{verbatim}
\begin{choices}
  \choice 9
  \choice 10
  \choice 11
  \choice 12
\end{choices}

\question Which of the following are true about exceptions in Python?
\begin{checkboxes}
  \choice An except block can only handle one type of exception at a time.
  \choice The except block is executed when an exception is raised in the try block.
  \choice The try/except block can be nested.
  \choice The try block can have multiple except blocks.
  \choice The try block can handle exceptions even if there is no except block.
  \choice none of these are true
\end{checkboxes}

\question Match each error to its meaning.
\begin{matching}
\mitem{1}{Syntax Error}{A}{a problem checked by the interpreter, incorrect use of the language itself}
\mitem{2}{Runtime Error}{B}{a problem that isn't checked by the interpreter but crashes the program}
\mitem{3}{Logic Error}{C}{a problem that doesn't crash the program, but produces unintended results}
\end{matching}

\question Which of the following are true of binary search? Select all that apply.
\begin{checkboxes}
  \choice Binary search compares the target value with each element sequentially.
  \choice Binary search divides a list into halves and eliminates one half at a time.
  \choice Binary search can only be applied to sorted lists or arrays.
  \choice Binary search has a worst-case time complexity of $O(\log n)$.
  \choice Binary search has a worst-case time complexity of $O(n)$.
  \choice Binary search can be implemented recursively or iteratively.
\end{checkboxes}

\question Match each code line to what it does.
\begin{matching}
\mitem{1}{\texttt{open(file, 'r')}}{A}{reads through the content of the file, error if the file does not exist}
\mitem{2}{\texttt{open(file, 'w')}}{B}{overwrites the content of the file, creates the file if it does not exist}
\mitem{3}{\texttt{open(file, 'a')}}{C}{writes from the end of the file; creates the file if it does not exist}
\mitem{4}{\texttt{open(file, 'x')}}{D}{creates the specified file, returns an error if the file exists}
\end{matching}

\question Which of the following are true of linear search? Select all that apply.
\begin{checkboxes}
  \choice Linear search has a time complexity of $O(n)$ in the worst case.
  \choice Linear search compares each element of the list with the target in order.
  \choice Linear search requires the list to be in descending order.
  \choice Linear search is faster than binary search on sorted lists.
  \choice Linear search has a best-case time complexity of $O(\log n)$.
  \choice Linear search is efficient for small datasets.
\end{checkboxes}

\question What will the list \texttt{[8,5,4,3,7,6,1,0,9,2]} look like after 4 swaps using each of the following methods? Match accordingly.
\begin{matching}
\mitem{1}{Quick Sort (using the last element as a pivot)}{A}{\texttt{[0,1,2,3,4,6,5,8,9,7]}}
\mitem{2}{Bubble Sort}{B}{\texttt{[5,4,3,7,8,6,1,0,9,2]}}
\mitem{3}{Selection Sort (selecting the largest element)}{C}{\texttt{[2,5,4,3,0,1,6,7,8,9]}}
\end{matching}

\question Which adjacency list represents the following graph?
\begin{center}
\end{center}
\begin{choices}
  \choice [[1, 5], [0, 2, 3, 5], [1, 4], [1, 4], [2, 3], [0, 1, 5]]
  \choice [[1, 5], [3, 5], [1, 4], [1, 4], [], [0, 5]]
  \choice [[1, 5], [3, 5], [1, 4], [1, 4], [3], [0]]
  \choice [[5], [0, 2, 3], [], [1], [2, 3], [0, 1, 5]]
\end{choices}

\question Which adjacency matrix represents the following graph?
\begin{center}
\end{center}
\begin{choices}
  \choice $\begin{bmatrix}0&1&0&1&0\\1&0&1&0&1\\0&1&0&1&0\\1&0&1&0&1\\0&1&0&1&0\end{bmatrix}$
  \choice $\begin{bmatrix}1&0&1&0&1\\0&1&0&1&0\\1&0&1&0&1\\0&1&0&1&0\\1&0&1&0&1\end{bmatrix}$
  \choice $\begin{bmatrix}0&1&1&1&0\\1&0&0&1&1\\1&0&1&0&1\\1&1&0&0&1\\0&1&1&1&0\end{bmatrix}$
  \choice $\begin{bmatrix}0&1&0&1&0\\1&0&1&0&1\\1&1&1&1&1\\1&0&1&0&1\\0&1&0&1&0\end{bmatrix}$
\end{choices}

\question What will happen if a recursive function does not have a base case?
\begin{choices}
  \choice The computer will violently explode.
  \choice The program will throw a syntax error.
  \choice The program will crash with a stack overflow.
  \choice The program will return an incorrect result (logic error).
\end{choices}

\question What will the following code output if the user enters \texttt{50}?
\begin{verbatim}
x = input("num:")
try:
    y = x * 2
    print(f"Result_1: {y}")
    z = int(x) + 2
    print(f"Result_2: {z}")
except ValueError:
    print("Error!")
\end{verbatim}
\begin{choices}
  \choice Result\_1: 100
Result\_2: 52
  \choice Result\_1: 5050
Result\_2: 52
  \choice Error!
  \choice Result\_1: 5050
Result\_2: Error!
\end{choices}

\question What will be printed when the following code runs?
\begin{verbatim}
a, b, c = 10, 25, 30
if a > b:
    if b > c:
        print("David")
    else:
        print("Mohammad")
else:
    if a + b > c:
        print("Eleena")
        if c - a == b:
            print("Jakob")
        else:
            print("Tiffany")
    else:
        print("Aayla")
\end{verbatim}
\begin{choices}
  \choice David
  \choice Mohammad
  \choice Jakob
  \choice Aayla
\end{choices}

\question What will the following code return?
\begin{verbatim}
aaa = 10
bbb = [5, 15]

def foo():
    aaa = 20
    bbb[0] = 50
    print(aaa)

def bar():
    global aaa
    aaa = 30
    bbb = [100, 200]
    return bbb

ccc = foo()
ddd = bar()
print(aaa, bbb, ccc, ddd)
\end{verbatim}
\begin{choices}
  \choice 20
30 [50, 15] None [100, 200]
  \choice 20
10 [50, 15] 20 [100, 200]
  \choice 10
10 [50, 15] None [5, 15]
  \choice 20
30 [100, 200] 20 [100, 200]
\end{choices}

\question Which of the following inputs $(x, y)$ will make this code output ``yippee''? Select all that apply.
\begin{verbatim}
x = int(input("num: "))
y = int(input("num: "))

if x > 0 and y > 0:
    print("yippee")
elif x < 0 and y < 0:
    print("no")
elif x > 0 or y > 0:
    if x == 0 or y == 0:
        print("no")
    else:
        print("yippee")
else:
    print("no")
\end{verbatim}
\begin{checkboxes}
  \choice (2, 4)
  \choice (2, -4)
  \choice (-2, 0)
  \choice (-2, -4)
  \choice (0, 0)
  \choice none of these
\end{checkboxes}

\question What will the following code output?
\begin{verbatim}
counter = 0
for i in range(6):
    if i % 2 == 0:
        continue
    for j in range(3):
        if j == 2:
            break
        counter += 1
        if i + j > 3:
            counter += 1
            break
print(counter)
\end{verbatim}
\begin{choices}
  \choice 5
  \choice 6
  \choice 7
  \choice 8
  \choice 9
  \choice the code will not execute OR infinite loop
\end{choices}

\question What will the following code output?
\begin{verbatim}
matrix = [
    [1, 2, 3, 4],
    [2, 1, 4, 3],
    [3, 4, 1, 2],
    [2, 3, 4, 1]
]
matrix[1].pop(2)
matrix[3].remove(matrix[2][2])
matrix.insert(2, [2, 4, 1, 3])
matrix[2].remove(3)
matrix.pop(1)
matrix[3].insert(1, matrix[2].pop(3))
print(matrix)
\end{verbatim}
\begin{choices}
  \choice [[2,4,1], [2,1], [4,4,2], [2,3,4,1]]
  \choice [[1,2,3,4], [2,4,1], [3,4,1], [2,2,3,4]]
  \choice [[1,2,3,4], [2,4,1], [3,4,1], [2,2,4,1]]
  \choice [[1,2,3,4], [2,4,1], [4,1,2], [2,3,4,1]]
\end{choices}

\question What will the following code output?
\begin{verbatim}
names = {
    1: ['Isa', 'Sandra', 'Alik'],
    2: ['Kyle', 'Nolan', 'Tinaye'],
    3: ['Simon', 'Jon-Luca', 'Cindy']
}

result = []
for key, value in names.items():
    if len(value[0]) % key == 0:
        result.append(value[1])
    elif key + len(value[0]) > 5:
        result.append(value[2])
print(result)
\end{verbatim}
\begin{choices}
  \choice ['Isa', 'Kyle', 'Nolan']
  \choice ['Jon-Luca', 'Sandra']
  \choice ['Sandra', 'Nolan', 'Cindy']
  \choice ['Alik', 'Tinaye', 'Jon-Luca']
\end{choices}

\question What will the following code output?
\begin{verbatim}
def fac(n):
    if n == 0:
        return 1
    return n * fac(n - 1)

def fib(n):
    if n <= 1:
        return n
    return fac(n - 1) + fib(n - 2)

print(fib(5))
\end{verbatim}
\begin{choices}
  \choice 7
  \choice 27
  \choice 48
  \choice 64
\end{choices}

\question What will the following code output?
\begin{verbatim}
try:
    data = {"a": 1, "b": 2}
    value = 10 / 0
    value = data["c"]
except ZeroDivisionError as e:
    print(f"oops, {e}")
except LookupError as e:
    print(f"oopsie: {e}")
except Exception as e:
    print(f"oops again: {e}")
\end{verbatim}
\begin{choices}
  \choice oops, division by zero
  \choice oops, division by zero
oopsie: 'c'
oops again: 'c'
  \choice oops, division by zero
oopsie: 'c'
oops again: division by zero
  \choice oops again: 'c'
\end{choices}

\question What should we replace line 9 with to make the code function as intended?
\begin{verbatim}
class Robot:
    def __init__(self, model, task):
        self.model = model
        self.task = task

    def perform_task(self):
        return f"Robot {self.model} is performing: {self.task}"

# i'm on break, fix it yourself
print(worker.perform_task())
\end{verbatim}
\begin{choices}
  \choice worker = new Robot("PhatGPT", "yapping")
  \choice worker = Robot("WALL-E", "garbage collection")
  \choice worker = new Robot("Sonny")
  \choice $worker = Robot.perform_task("GLaDOS", "experiment")$
\end{choices}

\end{questions}
\end{document}
