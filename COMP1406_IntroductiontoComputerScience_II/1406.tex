\documentclass[11pt,letterpaper]{exam}
\usepackage{../study_session}
\usepackage{graphicx}

\pagestyle{headandfoot}
\runningheadrule
\firstpageheadrule

\firstpageheader{COMP 1406Z}{Study Session Practice Questions}{\today}
\runningheader{COMP 1406Z}{Study Session Practice Questions}{Page \thepage\ of \numpages}
\runningfooter{}{}{}

\title{COMP 1406Z Introduction to Computer Science II}
\author{Study Session Questions}

\begin{document}
\maketitle
\vspace{1em}

\begin{questions}

\section*{Principles of OOP}

\question Which of the following are OOP principles? Select all that apply.
\begin{checkboxes}
\choice compilation
\choice encapsulation
\choice polymorphism
\choice abstraction
\choice inheritance
\choice debugging
\end{checkboxes}

\mcquestion{What is abstraction in Java?}{
\choice hiding the implementation details and showing only the functionality
\choice hiding the object state
\choice writing multiple methods with the same name
\choice using interfaces and classes
}

\mcquestion{What must a class do if it implements an interface?}{
\choice be marked as abstract
\choice implement all abstract methods of the interface
\choice explicitly extend the interface
\choice contain only static methods
}

\mcquestion{What is the main purpose of encapsulation in Java?}{
\choice To restrict direct access to object data and ensure controlled access.
\choice To provide methods to manipulate data directly.
\choice To use runtime polymorphism.
\choice To inherit properties from a parent class.
}
\newpage

\shortanswer{Place the four access modifiers in order, starting with the highest level of encapsulation.}

\mcquestion{What is the concept of inheritance in Java?}{
\choice A method of implementing interfaces for improved organization.
\choice A process for defining private methods and variables.
\choice A feature that ensures code executes in a specific order.
\choice A mechanism for one class to acquire the properties and methods of another.
}

\shortanswer{Which keyword is used to inherit a class in Java?}

\shortanswer{What operator can be used to check the type of an object?}

\mcquestion{What is the difference between an abstract class and an interface in Java?}{
\choice an interface shouldn't have attributes or constants, an abstract class can
\choice an abstract class supports multiple inheritance, an interface does not
\choice abstract classes can have methods with implementation, interfaces shouldn't
\choice interfaces do not support abstraction, whereas an abstract class does
}

\tfquestion{An abstract subclass can provide implementation of inherited abstract methods.}

\question If A is a subclass of B, which of the following are true? Select all that apply.
\begin{checkboxes}
\choice class B can access private members of class A
\choice objects of type A can be stored in variables of type B
\choice objects of type B can be stored in variables of type A
\choice class A inherits all non-private members of class B
\end{checkboxes}

\newpage

\mcquestion{Assume we are designing a BankAccount class. Which of the following methods are very likely to be static?}{
\choice \texttt{deposit()}
\choice \texttt{getBalance()}
\choice \texttt{convertToUSD()}
\choice \texttt{getExchangeRate()}
}

\question Which of the following are true of the final modifier? Select all that apply.
\begin{checkboxes}
\choice a variable declared final can't be reassigned after initialization
\choice a class declared as final cannot be subclassed
\choice a final method cannot be overridden by subclasses
\choice declaring a method as final will also make the class final
\choice a final class must have a final constructor
\choice a class marked as final must be abstract
\end{checkboxes}

\shortanswer[1.0in]{Match each of the following memory areas to their purpose.}

\shortanswer[0.75in]{Which keyword in Java is used to refer to the current instance of a class within its own methods or constructors?}

\question Which of the following are true of abstract classes? Select all that apply.
\begin{checkboxes}
\choice abstract classes are meant to be a framework for child classes
\choice abstract classes cannot be instantiated
\choice abstract classes cannot be marked final
\choice abstract classes cannot have a constructor
\end{checkboxes}

\section*{Polymorphism and ADTs}

\mcquestion{What is polymorphism in Java?}{
\choice the property of an object to inherit more than one superclass at once
\choice the ability of a method to modify its arguments directly
\choice the act of defining multiple variables of the same type
\choice the property of an object to take on multiple different forms
}

\question Which of the following are advantages of using polymorphism? Select all that apply.
\begin{checkboxes}
\choice simplifies code maintenance by promoting modularity
\choice enables a single interface to represent multiple behaviors
\choice improves code scalability and flexibility
\choice guarantees faster execution speed
\choice increased memory efficiency
\choice code is easier to understand
\end{checkboxes}

\question Is this an example of method overloading, or method overriding?
\begin{verbatim}
class Food {
    void prepare() {
        System.out.println("Preparing food");
    }
}

class Pasta extends Food {
    void prepare() {
        System.out.println("Mamma mia, pasta");
    }
}

public static void main(String[] args) {
    Food myFood = new Pasta();
    myFood.prepare();
}
\end{verbatim}
\begin{choices}
\choice method overloading
\choice method overriding
\choice both
\choice neither
\end{choices}

\question Is this an example of method overloading, or method overriding?
\begin{verbatim}
class Food {
    void prepare() {
        System.out.println("Preparing food...");
    }

    void prepare(String name) {
        System.out.println("Preparing " + name + "...");
    }
}

class Pizza extends Food {
    void prepare() {
        System.out.println("Preparing pizza...");
    }

    void prepare(String name, int size) {
        System.out.println("Preparing " + size + "-in " + name);
    }
}
\end{verbatim}
\begin{choices}
\choice method overloading
\choice method overriding
\choice both
\choice neither
\end{choices}

\mcquestion{What is an abstract data type (ADT)?}{
\choice a data type defined by its implementation details
\choice a data type defined by its behavior and operations, not its implementation
\choice a specific type of data structure optimized for memory efficiency
\choice a specific format for exporting and importing data between systems
}

\mcquestion{What is the key characteristic of a Queue ADT?}{
\choice Last In, First Out (LIFO)
\choice First In, First Out (FIFO)
\choice Random Access
\choice First Out, In Last (FOIL)
}

\mcquestion{What distinguishes a LinkedList from an array?}{
\choice a linked list uses dynamic memory allocation
\choice a linked list stores elements in a contiguous memory block
\choice a linked list allows direct access to any element
\choice a linked list has a fixed size
}

\mcquestion{What is the primary advantage of using a LinkedList over an array?}{
\choice faster random access to elements
\choice efficient insertion and deletion operations
\choice ability to store only unique elements
\choice it uses less memory than arrays
}

\shortanswer[0.75in]{What is the time complexity of searching for an element in a singly linked list?}

\section*{GUIs, JavaFX, and MVC}

\mcquestion{Which of the following is NOT a feature of JavaFX?}{
\choice Styling using CSS
\choice 2D and 3D graphics support
\choice Integration with modern Java IDEs
\choice Built-in database management system
}

\mcquestion{What is the primary purpose of the Stage class in JavaFX?}{
\choice to define the layout of UI elements
\choice to act as the main container for the JavaFX application window
\choice to represent the controller of the application
\choice to define the application's event listeners
}

\mcquestion{What is the purpose of an event handler in JavaFX?}{
\choice To create new UI components
\choice To manage user interactions like clicks and key presses
\choice To manage data binding between the model and the view
\choice To apply styling to JavaFX nodes
}

\shortanswer[0.75in]{Which JavaFX class is typically used as the main entry point for an application?}

\mcquestion{In a JavaFX application using MVC, what would typically trigger a change in the Model?}{
\choice direct interaction with the Model by the View
\choice a user action in the View, processed by the Controller
\choice automatic updates from the View to the Model
\choice the initialization of the Stage
}

\mcquestion{What type of event does the setOnAction() method handle?}{
\choice Mouse events only
\choice Keyboard events only
\choice Action events like button clicks
\choice Focus change events
}

\section*{Code Tracing}

\question {What will be the output of this code?}
\begin{verbatim}
class Book {
    String title;
    double price;

    Book(String title, double price) {
        this.title = title;
        this.price = price;
    }
}

public class Test {
    public static void main(String[] args) {
        Book book1 = new Book("Title1", 29.99);
        Book book2 = new Book("Title1", 29.99);
        System.out.print(book1 == book2);
        System.out.print(book1.title.equals(book2.title));
    }
}
\end{verbatim}
\begin{choices}
\choice falsefalse
\choice falsetrue
\choice truefalse
\choice truetrue
\end{choices}

\question {What will be the output of this code?}
\begin{verbatim}
class Course {
    void enroll(Course course) {
        System.out.println("Student enrols in course.");
    }
    void enroll(CompSci course) {
        System.out.println("Student enrols in CS course.");
    }
}
class CompSci extends Course {
    void enroll(Course course) {
        System.out.println("CS student enrols in course.");
    }
    void enroll(CompSci course) {
        System.out.println("CS student enrols in CS course.");
    }
}
public class Test {
    public static void main(String[] args) {
        Course c1 = new CompSci();
        Course c2 = new Course();
        c1.enroll(c2);
        c2.enroll(c1);
    }
}
\end{verbatim}
\begin{choices}
\choice CS student enrols in course.\newline Student enrols in course.
\choice Student enrols in course.\newline CS student enrols in course.
\choice Student enrols in CS course.\newline Student enrols in CS course.
\choice CS student enrols in course.\newline Student enrols in CS course.
\end{choices}

\question {What will be the output of this code?}
\begin{verbatim}
interface Club {
    void join();
}

class SportsClub implements Club {
    public void join() {
        System.out.println("Joining a sports club.");
    }
}

class MusicClub implements Club {
    public void join() {
        System.out.println("Joining a music club.");
    }
}

public class Test {
    public static void main(String[] args) {
        Club club = new SportsClub();
        MusicClub musicClub = (MusicClub) club;
        musicClub.join();
    }
}
\end{verbatim}
\begin{choices}
\choice Joining a sports club.\newline Joining a music club.
\choice Joining a music club.
\choice Joining a sports club.
\choice Exception in thread "main"
\end{choices}

\question {What is the output of the following code?}
\begin{verbatim}
import java.util.ArrayList;

public static void main(String[] args) {
    ArrayList<Integer> list = new ArrayList<>();
    list.add(1);
    list.add(2);
    list.remove(1);
    list.add(4);
    list.add(1, 2);
    System.out.println(list);
}
\end{verbatim}
\begin{choices}
\choice [1, 2, 4]
\choice [2, 2, 4]
\choice [1, 4, 1]
\choice [2, 4, 1]
\end{choices}

\question {What is the output of the following code?}
\begin{verbatim}
import java.util.HashMap;

public static void main(String[] args) {
    HashMap<String, String> map = new HashMap<>();
    map.put("K", "Khushpreet");
    map.put("J", "Jack");
    map.put("K", "Kareem");
    System.out.println(map.get("K") + " " + map.size());
}
\end{verbatim}
\begin{choices}
\choice Khushpreet 3
\choice Kareem 2
\choice Khushpreet 2
\choice Compilation Error
\end{choices}

\question {What is the output of the following code?}
\begin{verbatim}
public class Main {
    public static void main(String[] args) {
        try {
            String str = "who is passing the exam";
            str = null;
            System.out.print(str.length());
        } catch (ArithmeticException e) {
            System.out.print("Ishaan, ");
        } catch (NullPointerException e) {
            System.out.print("Aaryan, ");
        } catch (Exception e) {
            System.out.print("Anushka, ");
        } finally {
            System.out.print("Ahmad, ");
        }
    }
}
\end{verbatim}
\begin{choices}
\choice Ishaan, Aaryan, Anushka, Ahmad,
\choice Aaryan, Ahmad,
\choice Aaryan, Anushka, Ahmad
\choice Ahmad,
\end{choices}

\question {What is the output of the following code?}
\begin{verbatim}
import java.io.*;

public static void main(String[] args) throws IOException {
    File IO Exception = new File("classlist.txt");
    file.createNewFile();

    try (PrintWriter writer = new PrintWriter(new
        FileWriter(file))) {
        writer.write("Chi M\n");
        writer.write("Bliss I");
    }

    try (BufferedReader reader = new BufferedReader(new
        FileReader(file))) {
        System.out.println(reader.readLine());
        System.out.println(reader.readLine());
        System.out.print(reader.readLine());
    }
}
\end{verbatim}
\begin{choices}
\choice Chi M Bliss I null
\choice Chi M\newline Bliss I null
\choice Chi M\newline Bliss I\newline IOException
\choice IOException
\end{choices}

\question {[Lecture 04, 07, 09] What will be the output of this program?}
\begin{verbatim}
import java.util.*;

class ClassMark {}
class Quiz extends ClassMark {}
class Tutorial extends ClassMark {}

public class Main {
    public static void main(String[] args) {
        Set<ClassMark> marks = new HashSet<>();
        marks.add(new Quiz());
        marks.add(new Tutorial());
        marks.add(new Quiz());

        System.out.println(marks.size());
    }
}
\end{verbatim}
\begin{choices}
\choice 1
\choice 2
\choice 3
\choice Compilation Error
\end{choices}

\question {Which of the following lines cause an exception (i.e. are invalid) if left in? Select any that apply.}
\begin{verbatim}
class Building {}
class Residence extends Building {}
class Stormont extends Residence {}
class Herzberg extends Building {}
class Southam extends Building {}

public static void main(String[] args) {
    Residence res = new Residence();
    Building b1 = (Building) res;
    Stormont storm = new Stormont();
    Residence res2 = (Residence) storm;
    Southam southam = (Southam) res2;
    Herzberg herz = (Herzberg) b1;
}
\end{verbatim}
\begin{checkboxes}
\choice line 9
\choice line 10
\choice line 11
\choice line 12
\choice line 13
\choice line 14
\end{checkboxes}

\question {What should we replace line 19 with to achieve the intended functionality of this JavaFX program?}
\begin{verbatim}
public class Main extends Application {
    public static void main(String[] args) { launch(args); }

    private int counter = 0;
    public void start(Stage primaryStage) {
        Label label = new Label("Count: 0");
        Button button = new Button("Increase Count");

        button.setOnAction(e -> {
            counter++;

            // i'm on break, fix it yourself
        });

        VBox layout = new VBox(10);
        layout.getChildren().addAll(label, button);

        primaryStage.setScene(new Scene(layout, 200, 150));
        primaryStage.setTitle("Counter App");
        primaryStage.show();
    }
}
\end{verbatim}
\begin{choices}
\choice counter = label.getText()
\choice label = new Label("Count: " + counter)
\choice label.text = "Count" + counter
\choice label.setText("Count: " + counter)
\end{choices}


\end{questions}

\end{document}
