\documentclass[11pt,letterpaper]{exam}
\usepackage{../study_session}
\usepackage{amsfonts} 

\pagestyle{headandfoot}
\runningheadrule
\firstpageheadrule

\firstpageheader{MATH 1104}{Study Session Practice Questions}{\today}
\runningheader{MATH 1104}{Study Session Practice Questions}{Page \thepage\ of \numpages}
\runningfooter{}{}{}

\title{MATH 1104 Linear Algebra}
\author{Study Session Questions}

\begin{document}
\maketitle
\vspace{1em}

\section*{Learning Objectives}

\begin{itemize}
\item You should be confident solving linear systems using row reduction and the matrix equation $Ax=b$, and you should know what linear dependence and span are.

\item You should be confident applying matrix operations, including inverses, and you should know how to calculate determinants and use Cramer's rule.

\item You should be confident analyzing vector spaces by identifying subspaces, computing dimension/rank, and finding bases for a space.

\item You should be confident working with complex numbers and computing eigenvalues and eigenvectors to study diagonalizability.

\item You should be confident calculating inner products, applying projections, and implementing the Gram-Schmidt process for orthonormal bases.
\end{itemize}

\begin{questions}

\mcquestion{Find the general solution to the corresponding homogenous system of linear equations as defined below.
\[
\begin{aligned}
x_1 + x_2 - x_3 &= 1\\
2x_1 + 3x_2 - 4x_3 &= 4\\
x_1 + 2x_2 - 3x_3 &= 2
\end{aligned}
\]
}{
\choice no solutions
\choice $(x_1,x_2,x_3)=t(1,0,1)+s(0,1,-2)$
\choice $(x_1,x_2,x_3)=t(-1,2,1)$
\choice $(x_1,x_2,x_3)=(1,1,0)$
}

\question How can we know the size of the solution set of a linear system? Select all true statements.
\begin{checkboxes}
\choice An inconsistent system (i.e $\left[\,0\ 0\ 0\mid 3\,\right]$) has no solution
\choice If the system is square and $\det(A)=0$, the solution is unique
\choice If all columns in the coefficient matrix are pivots, the solution is unique
\choice If there are any free variables, the system has infinite solutions
\choice If the rank is less than the number of variables, there are no solutions
\choice A consistent system with fewer equations than variables has infinite sols.
\end{checkboxes}

\mcquestion{Suppose $b=x_1a_1+x_2a_2$ as defined below. What is the sum $x_1+x_2$?
\[
a_1=\begin{bmatrix}2\\1\\2\end{bmatrix},\quad
a_2=\begin{bmatrix}-1\\-1\\-2\end{bmatrix},\quad
b=\begin{bmatrix}9\\2\\4\end{bmatrix}
\]
}{
\choice 3
\choice 6
\choice 9
\choice 12
}

\question Given the vectors defined below, which of the following statements are true? Select all that apply.
\[
v_1=\begin{bmatrix}1\\0\\2\end{bmatrix},\quad
v_2=\begin{bmatrix}2\\-1\\1\end{bmatrix},\quad
v_3=\begin{bmatrix}-2\\0\\0\end{bmatrix}
\]
\begin{checkboxes}
\choice $(3,-1,3)\in \operatorname{Span}\{v_1,v_2\}$
\choice The zero vector is in $\operatorname{Span}\{v_1,v_3\}$
\choice $(0,1,2)\in \operatorname{Span}\{v_1,v_3\}$
\choice $(7,-12,4)\in \operatorname{Span}\{v_1,v_2,v_3\}$
\choice All three vectors are linearly DEPENDENT.
\choice All three vectors span $\mathbb{R}^3$.
\end{checkboxes}

\mcquestion{For what values $(k,h)$ does the system defined below have a unique solution?
\[
\begin{bmatrix}
1&2&1\\
0&3&4\\
2&1&k
\end{bmatrix}
\begin{bmatrix}
x\\y\\z
\end{bmatrix}
=
\begin{bmatrix}
h\\5\\7
\end{bmatrix}
\]
}{
\choice $k=2,\ h=-3$
\choice $k\ne -2$, for all $h\in\mathbb{R}$
\choice $k\ne -3$, for all $h\in\mathbb{R}$
\choice $k\ne 3,\ k\ne 2$
\choice $h=4$, for all $k\in\mathbb{R}$
\choice $k\ne 4$, for all $h\in\mathbb{R}$
}

\shortanswer[1.2in]{For what values of $m$ is the set $\{u,v\}$ linearly dependent? Type them in numerical order, comma and space separated.
\[
u=\begin{bmatrix}m\\1\end{bmatrix},\quad
v=\begin{bmatrix}2\\m+1\end{bmatrix}
\]
}

\question Consider the following augmented matrix of a system of linear equations below. Select all statements that are TRUE.
\[
\left[\begin{array}{ccc|c}
1&3&2&5\\
0&1&4&-1\\
0&0&a^2-4a&a-4
\end{array}\right]
\]
\begin{checkboxes}
\choice When $a=0$, there are infinitely many solutions
\choice When $a\ne 0$, the system is consistent
\choice When $a=2$, the system has no solutions
\choice When $a=2$, the columns of the coefficient matrix are linearly independent
\choice When $a=4$, there are infinitely many solutions
\choice When $a\ne 4$, the columns of the coefficient matrix are linearly independent
\end{checkboxes}

\mcquestion{Compute the original matrix $A$ given its inverse, $A^{-1}$, defined below.
\[
A^{-1}=\begin{bmatrix}
1&-2&5\\
0&1&-4\\
0&0&1
\end{bmatrix}
\]
}{
\choice $\begin{bmatrix}-1&2&-5\\0&-1&4\\0&0&-1\end{bmatrix}$
\choice $\begin{bmatrix}1&1&1\\0&1&1\\1&0&0\end{bmatrix}$
\choice $\begin{bmatrix}1&2&3\\0&1&4\\0&0&1\end{bmatrix}$
\choice $\begin{bmatrix}1&0&0\\4&1&0\\2&5&1\end{bmatrix}$
\choice the original matrix is the identity matrix
\choice the original matrix does not exist
}

\question Which of the following matrices are invertible? Select all that apply.
\begin{checkboxes}
\choice $\begin{bmatrix}0&0\\0&0\end{bmatrix}$
\choice $\begin{bmatrix}3&1&2\\6&2&4\\0&-1&-2\end{bmatrix}$
\choice $\begin{bmatrix}1&2\\3&4\end{bmatrix}$
\choice $\begin{bmatrix}0&1&1\\0&0&1\\1&0&0\end{bmatrix}$
\end{checkboxes}

\question Let $A$, $B$, and $C$ each be $3\times 3$ matrices. If $\det(A)=2$, $\det(B)=4$, and $\det(C)=8$, select all responses that must be true.
\begin{checkboxes}
\choice $\det(3A)=18$
\choice $\det(ABC)=64$
\choice $\det\left(B^{-1}C^{T}\right)=2$
\choice $\det\left((2B)^{-1}\right)=\frac{1}{8}$
\choice $\det\left[A^{-1}(B^{T})^{2}\right]=8$
\choice $\det(AI)=2$
\end{checkboxes}

\mcquestion{Let $A$ and $B$ be $3\times 3$ matrices as defined below. If $\det(A)=4$, determine $\det(B)$.
\[
A=\begin{bmatrix}
a&b&c\\
u&v&w\\
x&y&z
\end{bmatrix},\quad
B=\begin{bmatrix}
a&3b+5a&c\\
u&3v+5u&w\\
x&3y+5x&z
\end{bmatrix}
\]
}{
\choice $\det(B)=-4$
\choice $\det(B)=60$
\choice $\det(B)=4$
\choice $\det(B)=8$
\choice $\det(B)=12$
\choice $\det(B)=20$
}

\mcquestion{Let $A$ and $B$ be matrices as defined below. Find the matrix $X$ such that $3X-B=AX+2I$.
\[
A=\begin{bmatrix}1&1\\0&2\end{bmatrix},\quad
B=\begin{bmatrix}1&2\\1&0\end{bmatrix},\quad
3X+B=AX+2I
\]
}{
\choice $\begin{bmatrix}1&1\\1&1\end{bmatrix}$
\choice $\begin{bmatrix}2&1\\1&3\end{bmatrix}$
\choice $\begin{bmatrix}3&2\\1&2\end{bmatrix}$
\choice $\begin{bmatrix}2&2\\1&2\end{bmatrix}$
}

\mcquestion{Find all values of $k$ for which the matrix $A$ defined below is NOT invertible.
\[
A=\begin{bmatrix}
2&1&k\\
4&k&9\\
k&0&0
\end{bmatrix}
\]
}{
\choice $k=-3$
\choice $k=3$
\choice $k=-1$
\choice $k=1$
\choice $k=0$
\choice such a $k$ does not exist
}

\longanswer{How can Cramer's Rule be used to solve linear systems? Briefly explain in your own words.}

\mcquestion{What is the coordinate vector of $x=(5,0,4)$ relative to basis $B$ as defined below?
\[
B=\left\{
\begin{bmatrix}1\\0\\2\end{bmatrix},
\begin{bmatrix}0\\1\\1\end{bmatrix},
\begin{bmatrix}1\\1\\0\end{bmatrix}
\right\}
\]
}{
\choice $[x]_B=(5,3,-3)$
\choice $[x]_B=(3,-2,2)$
\choice $[x]_B=(1,-2,-4)$
\choice $[x]_B=(-5,0,-4)$
}

\question Which of the following statements about properties of transformations are TRUE? Select all that apply.
\begin{checkboxes}
\choice A linear map is invertible only if it is both one-to-one and onto
\choice The kernel is the set of all vectors that map to the zero vector
\choice The range of a transformation is the span of the columns of its matrix
\choice A transformation from $\mathbb{R}^n$ to $\mathbb{R}^m$ is always invertible if $n>m$
\choice Any set of independent vectors from the domain can form the standard matrix
\choice A transformation is one-to-one if its kernel contains only the zero vector
\end{checkboxes}

\mcquestion{Let $T:\mathbb{R}^4\to \mathbb{R}^3$ be the linear transformation as defined below. Select the most true response.
\[
T\left(\begin{bmatrix}x\\y\\z\\w\end{bmatrix}\right)
=
\begin{bmatrix}
x+3w\\
2x-y+z\\
4z-5w
\end{bmatrix}
\]
}{
\choice $T$ is one-to-one (injective).
\choice $T$ is onto (surjective).
\choice $T$ is neither injective nor surjective.
\choice $T$ is bijective, and thus, also invertible.
}

\question Let $A$ be a $6\times 9$ matrix such that row echelon form has 5 pivot positions. Which of these statements are TRUE?
\begin{checkboxes}
\choice $\dim(\operatorname{Nul}A)=4$
\choice $\operatorname{Rank}A=5$
\choice $\operatorname{Nul}A=\mathbb{R}^4$
\choice $\dim(\operatorname{Col}A)=5$
\end{checkboxes}

\mcquestion{Let matrix $A$ be as defined below. Which of the following sets is a basis for the null space of $A$?
\[
A=\begin{bmatrix}
1&0&-3&2\\
0&1&2&-4\\
0&0&0&0
\end{bmatrix}
\]
}{
\choice $\left\{\begin{bmatrix}3\\-2\\1\\0\end{bmatrix},\begin{bmatrix}-2\\4\\0\\1\end{bmatrix}\right\}$
\choice $\left\{\begin{bmatrix}1\\0\\0\end{bmatrix},\begin{bmatrix}0\\1\\0\end{bmatrix}\right\}$
\choice $\left\{\begin{bmatrix}3\\2\\0\\1\end{bmatrix},\begin{bmatrix}2\\-4\\1\\0\end{bmatrix}\right\}$
\choice $\left\{\begin{bmatrix}-3\\2\\1\\1\end{bmatrix}\right\}$
}

\question Select all of the sets below that are vector spaces.
\begin{checkboxes}
\choice $\left\{\begin{bmatrix}p\\q\\r\end{bmatrix}:\ 3p-4q=r,\ 2q=p+3q\right\}$
\choice $\left\{\begin{bmatrix}x\\y\\z\end{bmatrix}:\ x-2y+z=5\right\}$
\choice $\left\{\begin{bmatrix}3\\c\\d\end{bmatrix}:\ c,d\ \text{real}\right\}$
\choice $\left\{\begin{bmatrix}-a+2b\\a-2b\\3a-6b\end{bmatrix}:\ a,b\ \text{real}\right\}$
\end{checkboxes}

\mcquestion{Let $A$ be defined below and $w=(0,0,4,-2)$. Select the most correct response.
\[
A=\begin{bmatrix}
3&0&4&8\\
1&0&3&6\\
4&-2&2&4\\
4&1&0&0
\end{bmatrix}
\]
}{
\choice $w$ is in $\operatorname{Col}A$.
\choice $w$ is in $\operatorname{Nul}A$.
\choice $w$ is in both $\operatorname{Col}A$ and $\operatorname{Nul}A$.
\choice $w$ is in neither $\operatorname{Col}A$ nor $\operatorname{Nul}A$.
}

\mcquestion{Let $v=(0,1,1)$, an eigenvector of $A$. What is the corresponding eigenvalue?
\[
A=\begin{bmatrix}
4&0&0\\
0&2&1\\
0&1&2
\end{bmatrix}
\]
}{
\choice $-4$
\choice $4$
\choice $-3$
\choice $3$
\choice $-2$
\choice $2$
}

\mcquestion{What is the polar form of the complex number $z=3-3i$?}{
\choice $6(\cos(-\pi/4)-i\sin(-\pi/4))$
\choice $3\sqrt{2}(\cos(-\pi/4)+i\sin(-\pi/4))$
\choice $3\sqrt{2}(\cos(7\pi/4)-i\sin(7\pi/4))$
\choice $3\sqrt{2}(\sin(\pi/4)+i\cos(\pi/4))$
}

\question Which of the following equations involving complex numbers are true? Select all that apply.
\begin{checkboxes}
\choice The modulus of $3+4i$ is $5$.
\choice $(2i)^2=4i$
\choice $(2+3i)(2-3i)=13$
\choice $i^6=i$
\choice $\dfrac{3+4i}{1+2i}=i-1$
\choice $\dfrac{4+3i}{2-i}=5$
\end{checkboxes}

\mcquestion{Suppose $A$ is a $4\times 4$ NON-invertible matrix. Which of the following cannot be the characteristic polynomial of $A$?}{
\choice $\lambda^4(\lambda-2)$
\choice $-(\lambda+3)^2$
\choice $\lambda^2+4\lambda$
\choice $-\lambda(\lambda^3-5)$
}

\mcquestion{$\lambda=4$ is an eigenvalue for the matrix $B$ as defined below. Which one of the following sets is a basis for $E_4$?
\[
B=\begin{bmatrix}
6&-2&4\\
2&2&4\\
2&-2&8
\end{bmatrix}
\]
}{
\choice $\left\{\begin{bmatrix}1\\1\\-1\end{bmatrix},\begin{bmatrix}1\\1\\0\end{bmatrix}\right\}$
\choice $\left\{\begin{bmatrix}1\\1\\0\end{bmatrix},\begin{bmatrix}-2\\0\\1\end{bmatrix}\right\}$
\choice $\left\{\begin{bmatrix}2\\0\\2\end{bmatrix}\right\}$
\choice $\left\{\begin{bmatrix}-1\\1\\0\end{bmatrix}\right\}$
}

\mcquestion{Exactly one of the following statements involving diagonalizability is TRUE. Which one?}{
\choice Every invertible matrix is diagonalizable.
\choice Every diagonalizable matrix is invertible.
\choice All $n\times n$ diagonalizable matrices have $n$ distinct eigenvalues.
\choice All $n\times n$ diagonalizable matrices have $n$ linearly independent eigenvectors.
}

\mcquestion{Select the statement that is most true involving the diagonalizability of matrices $A$ and $B$.
\[
A=\begin{bmatrix}3&-1\\1&5\end{bmatrix},\quad
B=\begin{bmatrix}3&6\\4&1\end{bmatrix}
\]
}{
\choice $A$ is diagonalizable.
\choice $B$ is diagonalizable.
\choice Both $A$ and $B$ are diagonalizable.
\choice Neither $A$ nor $B$ are diagonalizable.
}

\question Let $u=(1,2,3)$ and $v=(3,2,1)$. Select all statements that are TRUE.
\begin{checkboxes}
\choice $u\cdot v=10$
\choice $\|u\|=\|v\|$
\choice The vectors $u$ and $v$ are orthogonal
\choice The sum $u+v$ is orthogonal to the difference $u-v$
\end{checkboxes}

\mcquestion{Determine the value $k$ for which $v=(k,5,7)$ lies in $W^\perp$ with $W$ as defined below.
\[
W=\operatorname{Span}\left\{
\begin{bmatrix}3\\4\\-2\end{bmatrix},
\begin{bmatrix}7\\0\\2\end{bmatrix}
\right\}
\]
}{
\choice $k=-1$
\choice $k=1$
\choice $k=-2$
\choice $k=2$
\choice any $k$
\choice no such $k$ exists
}

\mcquestion{Let $u=(2,1)$ and $v=(1,3)$. What is the angle between $u$ and $v$?}{
\choice $\pi/2$
\choice $\pi/3$
\choice $\pi/4$
\choice $\pi/6$
}

\mcquestion{Let $W=\operatorname{Span}\{(1,2,2)\}$ and $x=(2,3,5)$. What is the distance from $x$ to $W$?}{
\choice $3\sqrt{3}$
\choice $-\dfrac{7}{2}$
\choice $2\sqrt{5}$
\choice $2$
}

\longanswer{Briefly explain how to use the Gram-Schmidt process to turn linearly-independent vectors into an orthonormal basis.}

\mcquestion{Suppose the columns of $B$ form an orthogonal basis for $\mathbb{R}^4$. If $x=c_1b_1+c_2b_2+c_3b_3+c_4b_4$, what is the value of $c_2$?
\[
B=\begin{bmatrix}
1&2&4&1\\
2&-1&8&2\\
3&0&-4&9\\
4&0&-2&8
\end{bmatrix},\quad
x=\begin{bmatrix}7\\5\\-1\\4\end{bmatrix}
\]
}{
\choice $\dfrac{5}{4}$
\choice $-\dfrac{4}{25}$
\choice $\dfrac{9}{5}$
\choice $1$
}

\end{questions}

\end{document}